%%%%%%%%%%%%%%%%%%%%%%%%%%%%%%%%%%%%%%%%%%%%%%%%%%%%%%%%%%%%%%%%%%%%%%%%%%%%%%%%
% Author        : Evgenii Shiliaev (xshili00)
% Description   : First project in the Introduction to Game Development course.
%   It deals with an analysis of a selected title from the point of its genre, 
%   style, and mechanics.
%
% Repo Link     : https://github.com/Jekwwer/IZHV-Project01-2021
%%%%%%%%%%%%%%%%%%%%%%%%%%%%%%%%%%%%%%%%%%%%%%%%%%%%%%%%%%%%%%%%%%%%%%%%%%%%%%%%

\documentclass[a4paper,10pt, english]{article}

\usepackage[left=2.50cm,right=2.50cm,top=1.50cm,bottom=2.50cm]{geometry}
\usepackage[utf8]{inputenc}
\usepackage[T1]{fontenc}
\usepackage[czech]{babel}

\newcommand{\ph}[1]{\textit{[#1]}}

\title{Analysis of Mechanics}
\author{Evgenii Shiliaev (xshili00)}
\date{}

\begin{document}

\maketitle
\thispagestyle{empty}

{
    \large

    \begin{itemize}

        \item[] \textbf{Title:} \ph{Cities: Skylines }

        \item[] \textbf{Released:} \ph{2015}

        \item[] \textbf{Developer:} \ph{Colossal Order}

        \item[] \textbf{Primary Genre:} \ph{Simulator}

        \item[] \textbf{Secondary Genre:} \ph{City-builder, construction and management simulator}

        \item[] \textbf{Style:} \ph{Realistic}

    \end{itemize}

}

\section*{\centering Analysis}

\subsection*{Simulation \textgreater \ City-building}
The primary genre of the game is simulator, although someone might think it is a city-builder. This is a reasonable suggestion because, for the first hours, a player will do mainly zoning and road placement without extra attention to the city's characteristics and peoples' needs. As the city grows, a player has more tasks, such as controlling public transport and goods production, dividing the city into districts with policies, regulating citizens' education and the labor market, and many more. Cities: Skylines is a complex simulation with lots of micromanagement. That's why in later stages of the game, a player is more focused, for example, on traffic optimizing, not on new districts building.

\subsection*{Genres' reflection in the gameplay}

The primary genre reflects a wide-range simulation. There is a relief with water, energy, forest, and other resources, which should be used by the city efficiently. There live people that have their routines, routes, and favorite place to drink. The game controls everything from weather to traffic.

The city-building genre represents a physical interaction with the city. A player can change the relief, place a road, or zone a space as residential, for example. He can plot a new tram line or build a new park to boost the land price and tourism in the district. The city's appearance and effectiveness are affected primarily by the building.

The CMS (construction and management simulation) genre is the heart of the game economy and nonphysical interaction with the city. A player can increase/decrease taxes for a specific population group or allocate more/less budget for something. Also, he can manipulate some district policies to affect the peoples' lifestyle.

\subsection*{Genres interaction}

In my opinion, genres of the game interact very closely. The player's actions are manifested primarily in the simulation since the mechanics of other genres inextricably link with the game world. There are some examples of it:

\begin{itemize}
    \item We build a plant that pollutes nearby areas. If there is a source of water nearby, its' quality will reduce, and the incidence of diseases among the population will increase.

    \item Increasing the budget for the "public transport" category will increase the number of city vehicles on the streets and reduce car traffic around busy routes.
\end{itemize}

\subsection*{Style and gameplay}
Cities: Skylines has a realistic style with visuals and sounds of the true city. In my opinion, other ways of representing won't suit the complex city simulator where a player may need to cope with natural disasters, lack of jobs, or overly educated people.

\end{document}

% End of exe1.tex
